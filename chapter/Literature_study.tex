\chapter{Literature Study}
\label{chap:Literature Study}

The implementation of FDIR on satellites have multiple complications with regards to the type of data generated by a satellite and the methodologies that can be implemented within the time and memory constraint of a cube-sat processor.

%\section{Strategies for Satellite Constellations}
%\cite{Castel2006}
%
%\subsection{Individual Strategy}
%Each satellite performs it's own FDIR without communication with other satellites. Consequently, the satellite has its own algorithms and knowledge of when a fault occurs within it's own system and is only required to communicate with the ground station.
%
%\subsection{Centralised Strategies}
%Centralised strategies focus on the FDIR of the entire constellation performed by a single entity. This can either be one of the satellites or the ground station.
%\subsubsection{Master Centralised Strategy}
%
%\subsubsection{Opportunistic Centralised Strategy}
%
%\subsubsection{Global Centralised Strategy}
%
%\subsection{Mixed Strategy}
%
%\subsection{Distributed Strategies}
%\subsubsection{Common Distributed Strategy}
%
%\subsubsection{Individual Distributed Strategy}

\section{Anomaly Detection on Satellites}
Various methodologies have been tested on different component of satellites. Therefore a summary of these research articles are provided in this section.

\subsection{Analysis and Prediction of Satellite Anomalies}
%\cite{Wintoft}

\subsection{Agent-based algorithm for fault detection and recovery of gyroscope's drift in small satellite missions}
To ensure that the ADCS of satellites are autonomous every aspect of the control must be able to recover from faults. \cite{carvajal2017agent} developed an algorithm to evaluate the control of a gyroscope and detect whether drifting exists. If drifting is detected another algorithm is deployed to ensure the recovery of the gyroscope drift by updating the error state vector.

\textbf{Multivariate Anomaly Detection in Discrete and Continuous Telemetry Signals Using a Sparse Decomposition in a Dictionary}
\cite{Pilastre2020}

\textbf{Fault isolation of reaction wheels onboard three-axis controlled in-orbit satellite using ensemble machine learning}
\cite{rahimi2020fault}

\textbf{Fault tolerant control for satellites with four reaction wheels}
\cite{jin2008fault}

\textbf{Innovative Fault Detection, Isolation and Recovery Strategies On-Board Spacecraft: State of the Art and Research Challenges}
\cite{wander2013innovative}

\textbf{Machine learning methods for spacecraft telemetry mining}
\cite{ibrahim2018machine}

\textbf{Machine learning techniques for satellite fault diagnosis}
\cite{ibrahim2020machine}

\textbf{Satellite fault diagnosis using a bank of interacting Kalman filters}
\cite{Tudoroiu2007}

\textbf{A scheme of satellite multi-sensor fault-tolerant attitude estimation}
\cite{Zhou2016} implements a fault tolerant federated Kalman filter with three sub-filters for multi-sensor fault estimation. 

\textbf{Detection of satellite attitude sensor faults using the UKF}
\cite{Xiong2007} provides a fault detection method by using the residuals generated by an unscented Kalman filter to detect anomalies with a threshold based on a confidence level. 


\textbf{Sensor fault detection and recovery in satellite attitude control}
\cite{Nasrolahi2018} 

\textbf{Sensor Failure Detection in Dynamical Systems by Kalman Filtering Methodology}
While methods for sensor failure detection in other dynamical systems has also been developed which includes kalman filter methodology~\cite{Ciftciogl1991},

\textbf{Sensors Anomaly Detection of Industrial Internet of Things Based on Isolated Forest Algorithm and Data Compression}
isolation forests~\cite{Liu2021} and using LSTM on sensor data to detect anomalies on machines 

\textbf{LSTM-based Encoder-Decoder for Multi-sensor Anomaly Detection}
\cite{Malhotra2016}

\textbf{Sensor fault detection and isolation using adaptive extended Kalman filter}
\cite{van2012sensor}

\section{Statistical Methods}
\subsection{Pearson Correlation}
Vectors of certain sensors are highly correlated. For instance the vector of the earth sensor is highly correlated since the magnitude of the vector remains more or less constant. To detect anomalies the correlation of vectors can be measured and with a specified threshold the correlation can be indicated as a anomaly or nor.

The squared Pearson correlation coefficient (SPCC) for vectors depicted as
\linebreak
\\
\centerline{$a = [a_1, a_2, \ldots, a_L]^T,$}
\linebreak
\centerline{$b = [b_1, b_2, \ldots, b_L]^T,$}
\\
is defined as \cite{benesty2009pearson}
\begin{equation}
	\rho^2 (a,b) = \frac{E^2 (a,b)}{E(a^Ta)E(b^Tb)}.
\end{equation}
The correlation coefficient is proven to be constraint as
\begin{equation}
	0 \leq \rho \leq 1,
\end{equation}
where $\rho = 1$ is perfect linear correlation. 

\subsection{Variance}
Within a sequential data sample of the satellite, the variance of the variables should be within a given threshold if the satellite is in a stable condition. The variance of the data sample is defined as 
\begin{equation}
	S^2 = \frac{\sum(x_i + \bar{x})^2}{n-1}
\end{equation}
where $x$ defines the variable within the dataset.

\subsection{Kalman-Filter}
The Kalman-filter application would require the state-space matrices to be provided in the log file.

\subsection{Multivariate Guassian Distribution}
The assumption that the error of our data is generated with a Guassian distribution with a specific mean, $\mu$, and variance, $\sigma^2$, provides the opportunity for using multi-variate Gaussian distribution to determine the probability of a data-sample within a dataset. 
\begin{equation}
	\label{mean}
	\mu_j = \frac{1}{m} \sum_{i=1}^{m}x_j^{(i)}
\end{equation}

\begin{equation}
	\label{variance}
	\sigma_j^2 = \frac{1}{m} \sum_{i=1}^{m}(x_j^{(i)} - \mu_j)^2
\end{equation}

\begin{equation}
	\label{guassian distribution}
	p(x) = \prod_{j=1}^{n} \frac{1}{\sqrt{2\pi}\sigma_j}exp(-\frac{(x_j-\mu_j)^2}{2\sigma_j^2})
\end{equation}

For multi-variate Guassian distribution \cite{do2008multivariate}.

\begin{equation}
	\label{sum}
	\sum = \frac{1}{m}\sum_{i=1}^{m}(x^{(i)}-\mu)(x^{(i)}-\mu)^T
\end{equation}

\begin{equation}
	\label{multi-variate guassian distribution}
	p(x) = \frac{1}{(2\pi)^{\frac{n}{2}}{\lvert \sum \rvert}^\frac{1}{2}} exp(-\frac{1}{2}(x-\mu)^T{\sum}^{-1}(x-\mu))
\end{equation}

The Anomalies will be classified based on probabilities smaller than a given threshold $p(x) < \epsilon$.

\begin{algorithm}[!htb]
	\caption[Multi-variate Guassian Distribution]{Multi-variate Guassian Distribution Algorithm}
	\label{alg}
	\begin{algorithmic}[1]
		\State Determine feature vectors $x_i$
		\State Determine threshold probabilty, $\epsilon$
		\State Calculate $\mu_j$ with Eq~\ref{mean}
		\State Calculate $\sigma_j$ with Eq~\ref{variance}
		\State Calculate $p(x)$ with Eq~\ref{guassian distribution}
		\If{$p(x) < \epsilon$}
			\State Anomaly $= True$
		\Else
			\State Anomaly $= False$
		\EndIf
		
	\end{algorithmic}
\end{algorithm}

\subsection{Kullback-Leibler Divergence}
The Kullback-Leibler divergence quantifies the difference between two probability density functions, denoted as $p(x)$ and $q(x)$ \cite{hershey2007approximating}. Satellites are systems that are predictable within a time-series. The divergence between two sequential data buffers from the satellite will have a very similar probability distribution. Therefore calculating the difference between two datasets can be used to detect an anomaly based on a given threshold.

The difference between the probability distributions from datasets, $a$ and $b$, in Figure~\ref{Guassian plot} cannot simply be calculated as the difference in the mean or the difference in the variance. To overcome this, the divergence between the two distributions can be calculated. Intuitively a point $x$ with a high probability in the dataset $a$ should have a high probability in the dataset $b$ if the two datasets have a small divergence. 

\pgfmathdeclarefunction{gauss}{3}{%
	\pgfmathparse{1/(#3*sqrt(2*pi))*exp(-((#1-#2)^2)/(2*#3^2))}%
}
\begin{figure}[!h]
	\centering
	\textbf{Difference Between Probability Distributions}
	\begin{tikzpicture}
		\begin{axis}[
			no markers, 
			domain=-3:6, 
			samples=100,
			ymin=0,
			axis lines*=left, 
			xlabel=$x$,
			every axis y label/.style={at=(current axis.above origin),anchor=south},
			every axis x label/.style={at=(current axis.right of origin),anchor=west},
			height=5cm, 
			width=12cm,
			xtick=\empty, 
			ytick=\empty,
			enlargelimits=false, 
			clip=false, 
			axis on top,
			grid = major,
			hide y axis
			]
			
			\addplot [very thick,cyan!50!black] {gauss(x, 3, 1)};
			
			\pgfmathsetmacro\valueA{gauss(3,3,1)}
			\draw [gray] (axis cs:3,0) -- (axis cs:3,\valueA);
			
			\node[below] at (axis cs:3, 0)  {$\mu_p$}; 
			
			\addplot [very thick,red!50!black] {gauss(x, 1.5, 1.5)};
			
			\pgfmathsetmacro\valueB{gauss(1.5,1.5,1.5)}
			\draw [gray] (axis cs:1.5,0) -- (axis cs:1.5,\valueB);
			
			\node[below] at (axis cs:1.5, 0)  {$\mu_q$}; 
		\end{axis}
		
	\end{tikzpicture}
	\caption{Guassian Distributions}
	\label{Guassian plot}
\end{figure}

The divergence can be expressed as 

\begin{equation}
	KL(P\lvert\lvert Q) = \int p(x) \log \left( \frac{q(x)}{p(x)} \right)dx.
\end{equation}

\subsection{Canonical Correlation Analysis}
Due to the orbital nature of satellites there exist a correlation between various sensors. For instance the sun sensor, magnetometer and earth sensor are correlated based on the desired orientation and orbit of the satellite. This correlation might not be of linear nature, but with non-linear correlation methods such as kernel canonical correlation the correlation can be measured.

However, canonical correlation provides the measure of correlation between a multi-dimensional variable with another multi-dimensional variable. Although this seems profitable for satellite fault detection, it will only be applicable for each the comparison between individual sensors. This will indicate the non-linear correlation of the sun sensor with regards to the magnetometer. The problem however, according to \cite{chen2017fault} is to, determine the appropriate threshold for which to classify a fault. \cite{chen2017fault} proposed a method for determining the appropriate threshold on page 5, algorithm 1.
\cite{fukumizu2007statistical}
\cite{zhu2017quality}

Python - Pyrcca package

\subsubsection{K-means-based}
\subsubsection{Guassian Mixture Model}
\subsubsection{Just-In-Time-Learning}
\cite{chen2020just}

\section{Feature Extraction}
To 
https://towardsdatascience.com/feature-extraction-techniques-d619b56e31be
\subsection{Prony's Method}
\subsection{Convolutional Networks}
\subsection{K-means Clustering}
K-clustering: Clustering multiple points with similar features.
%\subsection{Principal Component Analysis}
%\cite{choi2005fault}
%\cite{ding2010application}
\subsection{Partial Least Square}
%\subsection{Independent Component Analysis}
\subsection{Locally Linear Embedding}
%\subsection{Linear Discriminant Analysis}
%\subsection{Autoencoder}
\subsection{t-Distributed Stochastic Neighbor Embedding}


\section{Supervised Learning}
Supervised learning consists of models that are trained on labelled data. This is not a problem with simulation, but with the real data, it is a problem and to provide tests on the real data to label it must be proficient. If unsupervised learning and statistical methods are not sufficient in their accuracy, a method for labelling the real data must be provided.

\subsection{Decision Trees}
For decision trees in the use of classification we use the Gini score

\begin{equation}
	G = \sum_{k=1}^{K} \hat{p}_{k} (1-\hat{p}_{k})
\end{equation}

\subsection{Random Forests}
\cite{Shi2006, Paul2018, Primartha2018}

\subsection{Long Short Term Memory}
Time-series data: LSTM or DLSTM

\subsection{Support Vector Machines}
Support Vector Machines

\subsection{Naive Bayes}
Naive Bayes

\subsection{K-nearest neighbours}
K-nearest neighbours

\subsection{Artificial Neural Networks}
Artificial Neural Networks

\section{Unsupervised Learning}
Density-based, distance, Clustering

\subsection{Kernel Adaptive Density-based}
Kernel adaptive density-based: Is an algorithm that uses the density factor of a data point relative to other data points to determine whether the data point is an outlier or not.

\subsection{Loda}
Loda: Is a fast and efficient anomaly detection algorithm that used histograms to evaluate data points to determine whether a data point is an outlier. Loda is an on-line method and not a batch method.

\subsection{Robust-kernel Density Estimation}
Robust-kernel density estimation

\section{Reinforcement Learning}
Active Anomaly detection with meta-policy (Meta-AAD) is a deep reinforcement learning approach that is based on the actor-critic model. The agent must query data points within the given dataset (where the queried point is the data top 1 data point). The query is given to a human 

\section{Summary}

