\chapter{Literature Study}
\label{chap:Literature Study}

The implementation of FDIR on satellites have multiple complications with regards to the type of data generated by a satellite and the methodologies that can be implemented within the time and memory constraint of a cube-sat processor.

\textbf{TODO: Research within the larger field of FDIR diagram}

\section{Anomaly Detection on Satellites}
Various methodologies have been tested on different component of satellites. Therefore a summary of these research articles are provided in this section.

\subsection{Analysis and Prediction of Satellite Anomalies}
%\cite{Wintoft}

\subsection{Agent-based algorithm for fault detection and recovery of gyroscope's drift in small satellite missions}
To ensure that the ADCS of satellites are autonomous every aspect of the control must be able to recover from faults. \cite{carvajal2017agent} developed an algorithm to evaluate the control of a gyroscope and detect whether drifting exists. If drifting is detected another algorithm is deployed to ensure the recovery of the gyroscope drift by updating the error state vector.

\textbf{Multivariate Anomaly Detection in Discrete and Continuous Telemetry Signals Using a Sparse Decomposition in a Dictionary}
\cite{Pilastre2020}

\textbf{Fault isolation of reaction wheels onboard three-axis controlled in-orbit satellite using ensemble machine learning}
\cite{rahimi2020fault}

\textbf{Fault tolerant control for satellites with four reaction wheels}
\cite{jin2008fault}

\textbf{Innovative Fault Detection, Isolation and Recovery Strategies On-Board Spacecraft: State of the Art and Research Challenges}
\cite{wander2013innovative}

\textbf{Machine learning methods for spacecraft telemetry mining}
\cite{ibrahim2018machine}

\textbf{Machine learning techniques for satellite fault diagnosis}
\cite{ibrahim2020machine}

\textbf{Satellite fault diagnosis using a bank of interacting Kalman filters}
\cite{Tudoroiu2007}

\textbf{A scheme of satellite multi-sensor fault-tolerant attitude estimation}
\cite{Zhou2016} implements a fault tolerant federated Kalman filter with three sub-filters for multi-sensor fault estimation. 

\textbf{Detection of satellite attitude sensor faults using the UKF}
\cite{Xiong2007} provides a fault detection method by using the residuals generated by an unscented Kalman filter to detect anomalies with a threshold based on a confidence level. 


\textbf{Sensor fault detection and recovery in satellite attitude control}
\cite{Nasrolahi2018} 

\textbf{Sensor Failure Detection in Dynamical Systems by Kalman Filtering Methodology}
While methods for sensor failure detection in other dynamical systems has also been developed which includes kalman filter methodology~\cite{Ciftciogl1991},

\textbf{Sensors Anomaly Detection of Industrial Internet of Things Based on Isolated Forest Algorithm and Data Compression}
isolation forests~\cite{Liu2021} and using LSTM on sensor data to detect anomalies on machines 

\textbf{LSTM-based Encoder-Decoder for Multi-sensor Anomaly Detection}
\cite{Malhotra2016}

\textbf{Sensor fault detection and isolation using adaptive extended Kalman filter}
\cite{van2012sensor}

%\section{Statistical Methods}

%
%\subsubsection{K-means-based}
%\subsubsection{Guassian Mixture Model}
%\subsubsection{Just-In-Time-Learning}
%\cite{chen2020just}
%
%\section{Feature Extraction}
%To 
%https://towardsdatascience.com/feature-extraction-techniques-d619b56e31be
%\subsection{Prony's Method}
%\subsection{Convolutional Networks}
%\subsection{K-means Clustering}
%K-clustering: Clustering multiple points with similar features.
%%\subsection{Principal Component Analysis}
%%\cite{choi2005fault}
%%\cite{ding2010application}
%\subsection{Partial Least Square}
%%\subsection{Independent Component Analysis}
%\subsection{Locally Linear Embedding}
%%\subsection{Linear Discriminant Analysis}
%%\subsection{Autoencoder}
%\subsection{t-Distributed Stochastic Neighbor Embedding}
%
%
%\section{Supervised Learning}
%Supervised learning consists of models that are trained on labelled data. This is not a problem with simulation, but with the real data, it is a problem and to provide tests on the real data to label it must be proficient. If unsupervised learning and statistical methods are not sufficient in their accuracy, a method for labelling the real data must be provided.
%
%\subsection{Decision Trees}
%For decision trees in the use of classification we use the Gini score
%
%\begin{equation}
%	G = \sum_{k=1}^{K} \hat{p}_{k} (1-\hat{p}_{k})
%\end{equation}
%
%\subsection{Random Forests}
%\cite{Shi2006, Paul2018, Primartha2018}
%
%\subsection{Long Short Term Memory}
%Time-series data: LSTM or DLSTM
%
%\subsection{Support Vector Machines}
%Support Vector Machines
%
%\subsection{Naive Bayes}
%Naive Bayes
%
%\subsection{K-nearest neighbours}
%K-nearest neighbours
%
%\subsection{Artificial Neural Networks}
%Artificial Neural Networks
%
%\section{Unsupervised Learning}
%Density-based, distance, Clustering
%
%\subsection{Kernel Adaptive Density-based}
%Kernel adaptive density-based: Is an algorithm that uses the density factor of a data point relative to other data points to determine whether the data point is an outlier or not.
%
%\subsection{Loda}
%Loda: Is a fast and efficient anomaly detection algorithm that used histograms to evaluate data points to determine whether a data point is an outlier. Loda is an on-line method and not a batch method.
%
%\subsection{Robust-kernel Density Estimation}
%Robust-kernel density estimation
%
%\section{Reinforcement Learning}
%Active Anomaly detection with meta-policy (Meta-AAD) is a deep reinforcement learning approach that is based on the actor-critic model. The agent must query data points within the given dataset (where the queried point is the data top 1 data point). The query is given to a human 

\section{Summary}

