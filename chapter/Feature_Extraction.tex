\chapter{Feature Extraction}
\label{chap:Feature Extraction}


\section{Binary Feature Extraction}
The classes hyperparameter will be set to 2. The only problem with this, is a very evident 2 classes is eclipse and non-eclipse. Consequently each feature extraction model will be trained twice. Once for the eclipse and once for the non-eclipse period. The features extracted will be used with the detection algorithms.
%\subsection{Linear Regression}

\subsection{Local Outlier Factor}
\label{section:OutlierFactor}
Most algorithms for anomaly detection are based on a metric which accounts for the entire dataset~\cite{breunig2000lof}. However, many anomalies are identifiable in relation to the local neighbourhood of data points and not the overall dataset. Therefore, \cite{breunig2000lof} developed the local outlier factor \(LOF\) algorithm that provides a measure of a data point's "outlierness". This implies that a data point is not classified as an anomaly or not, but a local outlier factor is calculated to determine how much a data point is distantiated from it's $k$-nearest neighbours. This is clearly demonstrated in Figure~\ref{Figure-LOF_measure} where the data points which are clustered together have smaller LOF's than data points which are removed from the highly dense areas.

\begin{figure}[!hbt]
	\centering
	\import{Figures/}{LOF.pgf}
	\caption{Plane perpendicular to $\mathbf{R}_{SE}$ and at center of earth}
	\label{fig:localOutlierFactor}
\end{figure}

To calculate the LOF, the $k$-distance must be calculated and also the local reachability density \(lrd\). The $k$-distance, is the $k^{th}$ ranked $distance(o,p_i)$. Where $distance(o,p_i)$ is the distance between data point $o$ and any data point $p_i$, with $i \in N$, where $N$ is the number of data points within the dataset with a minimum value of $MinPts$. To reduce fluctuations in the $distance(o,p_i)$ the distance between $o$ and $p_i$ is replaced with 
\begin{equation}
max \{distance(o,p_i), k\text{-distance}\} 
\end{equation}
and will henceforth be referred to as the reachability distance~\cite{breunig2000lof}. The $lrd$ of a data point, $p$, is calculated as 
\begin{equation}
lrd_{MinPts}(p) = 1/\left(\frac{\sum\limits_{o \in N_{MinPts}(p)}^{} reach\-dist_{MinPts}(p,o)}{|N_{MinPts}(p)|}\right)
\label{Eq-lrd}
\end{equation}
and denotes "the inverse of the average reachability distance based on the $MinPts$-nearest neighbours of the $p$" --- \cite{breunig2000lof}. Eq~\ref{Eq-lrd} enables the calculation for the $LOF$ of point $p$ as shown in Eq~\ref{Eq-LOF}
\begin{equation}
LOF_{MinPts}(p) = \frac{\sum\limits_{o \in N_{MinPts}(p)}^{}\frac{lrd_{MinPts}(o)}{lrd_{MinPts}(p)}}{|N_{MinPts}(p)|}
\label{Eq-LOF}
\end{equation}
The rule of thumb for detecting an outlier is that when the LOF is larger than 1, then the point is considered an outlier with respect to its neighbourhood. This however is not fixed and the threshold can be changed depending on the application.
This method is aimed at producing a measure of the "outlierness" of a data point within a local neighbourhood and not for all the data points. This method will thus be implemented for the satellite anomaly detection, since it will detect anomalies within the two neighbourhoods produced by the eclipse during orbit. This method will also be able to detect measurements of earth sensors, sun sensors and magnetometers that drastically change from the previous orbital data. 
%For example in Fig~\ref{Figure-Satellite_orbit} it is evident that the LOF will be comparatively larger for the red data points, which are anomalies, to the blue data points that are the normal orbit of the satellite.

%\begin{figure}[h!tb]
%	\centering
%	\begin{tikzpicture}
%	\begin{axis}[title = Earth Sensor During Multiple Orbits]
%	\addplot3 table [x =Earth x, y = Earth y, z=Earth z, col sep=comma, only marks, scatter, blue, fill opacity=0.1, draw opacity = 0]{Data/3D_orbit.csv};
%	\addplot3 table [x =Earth_anomaly x, y = Earth_anomaly y, z=Earth_anomaly z, col sep=comma, only marks, scatter, red, fill opacity=0.1, draw opacity = 0]{Data/3D_orbit.csv};
%	\end{axis}
%	\label{Figure-Satellite_orbit}
%	\end{tikzpicture}
%\end{figure}
%\subsection{K-means Clustering}
%
%\subsection{Prony's Method}
%
%\subsection{Partial Least Square}
%
%\subsection{t-Distributed Stochastic Neighbor Embedding}

\subsection{Dynamic Mode Decomposition}
The proposed method by \cite{DeSilva2020} uses Dynamic Mode Decomposition (DMD), which was initially developed by \cite{schmid2011applications} and further expanded to include control by \cite{proctor2016dynamic}, to provide an estimation of a sensor vector based on the previous measurement of the sensor as well as the measurements of the other sensors in the system. DMD was first developed in the fluids community and constructed a matrix $\mathbf{A}$ to relate the state vector $x$ with the following time step of the state vector, $x_{k+1}$. The state vector, in our case, will be the measurement vector of the specific sensor that we want to monitor.
\begin{equation}
	\mathbf{x}_{k+1} = \mathbf{Ax}_k
\end{equation}
Where $\mathbf{x}_k$ and $\mathbf{x}_{k+1}$ during a specified number of time steps, will be denoted as $\mathbf{X}$ and $\mathbf{X'}$ respectively.

The method of DMD, however, is useful for high order systems where the calculation of $\mathbf{A}$ is computationally intensive. This is not the case for our system, and using DMD is not justifiable and consequently, a linear regression model is implemented. Therefore with the pseudo-inverse of $\mathbf{X}$, denoted as $\mathbf{X^{\dagger}}$, we calculate $\mathbf{A}$ as
\begin{equation}
	\mathbf{A} = \mathbf{X}\mathbf{X^{\dagger}}
\end{equation}
This necessitates data of the state vector over time. The article by \cite{DeSilva2020} however includes $\mathbf{B}$ to relate the vector measurements of the other sensors to adjust the predicted state, $X_{k+1}$ of the monitored sensor. 
\begin{equation}
	\mathbf{X}_{k+1} = \mathbf{AX}_k + \mathbf{BY}_k
	\label{control DMD}
\end{equation}
Where $\mathbf{Y}_k$ is the other sensor measurements, this is adjusted for our use case, where $\mathbf{Y}_k$ is the control torques for the magnetorquers and reaction wheels, while $\mathbf{X}_k$ is all of the sensor measurements. Consequently, the model of Eq~\ref{control DMD} denotes the prediction of the sensor measurements at time step $k+1$ based on the current sensor measurements and control inputs.
Thereafter, as implemented by \cite{DeSilva2020} the model is adjusted with a Kalman Filter. From $\mathbf{A}$ and $\mathbf{B}$ the Kalman filter can be implemented to predict $\mathbf{X}_{k+1}$
\begin{equation}
	\hat{\mathbf{X}}_{k+1} = \mathbf{A}\hat{\mathbf{X}}_k + \mathbf{B}\mathbf{Y}_k + K(\mathbf{X}_k - \hat{\mathbf{X}}_k)
\end{equation}
where $K = 0.001$. After the calculation of $\hat{\mathbf{X}}_{k+1}$ \cite{DeSilva2020} proposes a moving average of the innovation covariance
\begin{equation}
	\mathbf{V}_k = \frac{1}{N} \sum_{i=k-N}^k (\mathbf{X}_i - \hat{\mathbf{X}}_i)(\mathbf{X}_i - \hat{\mathbf{X}}_i)^T
\end{equation}
where $N$ is the number of timesteps to account for. The moving average is used as an additional input parameter for the classification of anomalies based on $\mathbf{X}$.

\textbf{TODO: Plot sum of moving average for normal and anomaly}

\section{Summary}