\chapter{Statistical Background}
\label{chap:StatsBack}

This chapter provides statistical background for the understanding of both the EKF and the various classification techniques discussed in Chapter~\ref{chap:Detection} and Chapter~\ref{chap:Isolation}.

\section{Mean and Expected Value}
The mean
\begin{equation}
\label{mean}
\mu_j = \frac{1}{m} \sum_{i=1}^{m}x_j^{(i)}
\end{equation}

\section{Variance}
\begin{equation}
\label{variance}
\sigma_j^2 = \frac{1}{m} \sum_{i=1}^{m}(x_j^{(i)} - \mu_j)^2
\end{equation}

\section{Probability distribution}
\textbf{TODO: Normal distribution plot}
\begin{equation}
\label{guassian distribution}
p(x) = \prod_{j=1}^{n} \frac{1}{\sqrt{2\pi}\sigma_j}exp(-\frac{(x_j-\mu_j)^2}{2\sigma_j^2})
\end{equation}

\section{Covariance}

\section{Pearson Correlation}
Vectors of certain sensors are highly correlated. For instance the vector of the earth sensor is highly correlated since the magnitude of the vector remains more or less constant. To detect anomalies the correlation of vectors can be measured and with a specified threshold the correlation can be indicated as a anomaly or nor.

The squared Pearson correlation coefficient (SPCC) for vectors depicted as
\linebreak
\\
\centerline{$a = [a_1, a_2, \ldots, a_L]^T,$}
\linebreak
\centerline{$b = [b_1, b_2, \ldots, b_L]^T,$}
\\
is defined as \cite{benesty2009pearson}
\begin{equation}
	\rho^2 (a,b) = \frac{E^2 (a,b)}{E(a^Ta)E(b^Tb)}.
\end{equation}
The correlation coefficient is proven to be constraint as
\begin{equation}
	0 \leq \rho \leq 1,
\end{equation}
where $\rho = 1$ is perfect linear correlation. 

\section{Multivariate Guassian Distribution}
The assumption that the error of our data is generated with a Guassian distribution with a specific mean, $\mu$, and variance, $\sigma^2$, provides the opportunity for using multi-variate Gaussian distribution to determine the probability of a data-sample within a dataset. 

For multi-variate Guassian distribution \cite{do2008multivariate}.

\begin{equation}
	\label{sum}
	\sum = \frac{1}{m}\sum_{i=1}^{m}(x^{(i)}-\mu)(x^{(i)}-\mu)^T
\end{equation}

\begin{equation}
	\label{multi-variate guassian distribution}
	p(x) = \frac{1}{(2\pi)^{\frac{n}{2}}{\lvert \sum \rvert}^\frac{1}{2}} exp(-\frac{1}{2}(x-\mu)^T{\sum}^{-1}(x-\mu))
\end{equation}

The Anomalies will be classified based on probabilities smaller than a given threshold $p(x) < \epsilon$.

\begin{algorithm}[!htb]
	\caption[Multi-variate Guassian Distribution]{Multi-variate Guassian Distribution Algorithm}
	\label{alg}
	\begin{algorithmic}[1]
		\State Determine feature vectors $x_i$
		\State Determine threshold probabilty, $\epsilon$
		\State Calculate $\mu_j$ with Eq~\ref{mean}
		\State Calculate $\sigma_j$ with Eq~\ref{variance}
		\State Calculate $p(x)$ with Eq~\ref{guassian distribution}
		\If{$p(x) < \epsilon$}
			\State Anomaly $= True$
		\Else
			\State Anomaly $= False$
		\EndIf
		
	\end{algorithmic}
\end{algorithm}

\section{Kullback-Leibler Divergence}
The Kullback-Leibler divergence quantifies the difference between two probability density functions, denoted as $p(x)$ and $q(x)$ \cite{hershey2007approximating}. Satellites are systems that are predictable within a time-series. The divergence between two sequential data buffers from the satellite will have a very similar probability distribution. Therefore calculating the difference between two datasets can be used to detect an anomaly based on a given threshold.

The difference between the probability distributions from datasets, $a$ and $b$, in Figure~\ref{Guassian plot} cannot simply be calculated as the difference in the mean or the difference in the variance. To overcome this, the divergence between the two distributions can be calculated. Intuitively a point $x$ with a high probability in the dataset $a$ should have a high probability in the dataset $b$ if the two datasets have a small divergence. 


%\begin{figure}[!h]
%	\centering
%	\textbf{Difference Between Probability Distributions}
%	\begin{tikzpicture}
%		\begin{axis}[
%			no markers, 
%			domain=-3:6, 
%			samples=100,
%			ymin=0,
%			axis lines*=left, 
%			xlabel=$x$,
%			every axis y label/.style={at=(current axis.above origin),anchor=south},
%			every axis x label/.style={at=(current axis.right of origin),anchor=west},
%			height=5cm, 
%			width=12cm,
%			xtick=\empty, 
%			ytick=\empty,
%			enlargelimits=false, 
%			clip=false, 
%			axis on top,
%			grid = major,
%			hide y axis
%			]
%			
%			\addplot [very thick,cyan!50!black] {gauss(x, 3, 1)};
%			
%			\pgfmathsetmacro\valueA{gauss(3,3,1)}
%			\draw [gray] (axis cs:3,0) -- (axis cs:3,\valueA);
%			
%			\node[below] at (axis cs:3, 0)  {$\mu_p$}; 
%			
%			\addplot [very thick,red!50!black] {gauss(x, 1.5, 1.5)};
%			
%			\pgfmathsetmacro\valueB{gauss(1.5,1.5,1.5)}
%			\draw [gray] (axis cs:1.5,0) -- (axis cs:1.5,\valueB);
%			
%			\node[below] at (axis cs:1.5, 0)  {$\mu_q$}; 
%		\end{axis}
%		
%	\end{tikzpicture}
%	\caption{Guassian Distributions}
%	\label{Guassian plot}
%\end{figure}

The divergence can be expressed as 

\begin{equation}
	KL(P\lvert\lvert Q) = \int p(x) \log \left( \frac{q(x)}{p(x)} \right)dx.
\end{equation}

\section{Canonical Correlation Analysis}
Due to the orbital nature of satellites there exist a correlation between various sensors. For instance the sun sensor, magnetometer and earth sensor are correlated based on the desired orientation and orbit of the satellite. This correlation might not be of linear nature, but with non-linear correlation methods such as kernel canonical correlation the correlation can be measured.

However, canonical correlation provides the measure of correlation between a multi-dimensional variable with another multi-dimensional variable. Although this seems profitable for satellite fault detection, it will only be applicable for each the comparison between individual sensors. This will indicate the non-linear correlation of the sun sensor with regards to the magnetometer. The problem however, according to \cite{chen2017fault} is to, determine the appropriate threshold for which to classify a fault. \cite{chen2017fault} proposed a method for determining the appropriate threshold on page 5, algorithm 1.
\cite{fukumizu2007statistical}
\cite{zhu2017quality}

\section{Kalman-Filter}
The Kalman-filter application would require the state-space matrices to be provided in the log file.
