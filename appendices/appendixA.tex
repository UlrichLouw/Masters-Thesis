This section will show the derivation of the Jacobian matrix $\mathbf{F}_{t}$ that must be constructed in the execution of the full state EKF, as described in Section 5.4.

The continuous system perturbation matrix $\mathbf{F}_{t}$ can be constructed by determining its individual components, thus
$$
\mathbf{F}_{t}=\left[\begin{array}{ll}
\frac{\partial \dot{\omega}_{B}^{I}}{\partial \omega_{B}^{I}} & \frac{\partial \dot{\omega}_{B}^{I}}{\partial \mathbf{q}} \\
\frac{\partial \dot{\mathbf{q}}}{\partial \omega_{B}^{I}} & \frac{\partial \dot{\mathbf{q}}}{\partial \mathbf{q}}
\end{array}\right]_{\omega_{B}^{I}=\hat{\omega}_{B}^{I}, \mathbf{q}=\hat{\mathbf{q}}}
$$
Note that the subscript ' $t$ ' indicating the time domain has been dropped from Equation $5.4 .5$ to simplify the derivation. The non-linear function $\boldsymbol{f}(\mathbf{x})$ can be separated into two parts: a non-linear function describing $\dot{\omega}_{B}^{I}$ and a non-linear function describing $\dot{\mathbf{q}}$. The continuous non-linear system equation with regards to $\dot{\omega}_{B}^{I}$ is the Euler dynamic equation, or
$$
\dot{\boldsymbol{\omega}}_{B}^{I}=\mathbf{J}^{-1}\left(\mathbf{N}_{c}+\mathbf{N}_{d}-\boldsymbol{\omega}_{B}^{I} \times\left(\mathbf{J} \boldsymbol{\omega}_{B}^{I}+\mathbf{h}_{w}\right)\right)
$$
The individual components of Equation B.2 can also be expressed as
$$
\begin{aligned}
\dot{\omega}_{x i} &=\frac{1}{I_{x x}}\left(N_{c x}+N_{d x}-\omega_{y i}\left(I_{z z} \omega_{z i}+h_{z}\right)+\omega_{z i}\left(I_{y y} \omega_{y i}+h_{y}\right)\right) \\
\dot{\omega}_{y i} &=\frac{1}{I_{y y}}\left(N_{c y}+N_{d y}-\omega_{z i}\left(I_{x x} \omega_{x i}+h_{x}\right)+\omega_{x i}\left(I_{z z} \omega_{z i}+h_{z}\right)\right) \\
\dot{\omega}_{z i} &=\frac{1}{I_{z z}}\left(N_{c z}+N_{d z}-\omega_{x i}\left(I_{y y} \omega_{y i}+h_{y}\right)+\omega_{y i}\left(I_{x x} \omega_{x i}+h_{x}\right)\right) .
\end{aligned}
$$
Using Equations B.3 to B.5, $\frac{\partial \dot{\omega}_{B}^{I}}{\partial \omega_{B}^{I}}$ can be determined by taking each individual partial derivative, which delivers
$$
\frac{\partial \dot{\boldsymbol{\omega}}_{B}^{I}}{\partial \boldsymbol{\omega}_{B}^{I}}=\left[\begin{array}{ccc}
0 & \frac{\omega_{z i}\left(I_{y y}-I_{z z}\right)-h_{z}}{I_{x x}} & \frac{\omega_{y i}\left(I_{y y}-I_{z z}\right)+h_{y}}{I_{x x}} \\
\frac{\omega_{z i}\left(I_{z z}-I_{x x}\right)+h_{z}}{I_{y y}} & 0 & \frac{\omega_{x i}\left(I_{z z}-I_{x x}\right)-h_{x}}{I_{y y}} \\
\frac{\omega_{y i}\left(I_{x x}-I_{y y}\right)-h_{y}}{I_{z z}} & \frac{\omega_{x i}\left(I_{x x}-I_{y y}\right)+h_{x}}{I_{z z}} & 0
\end{array}\right] .
$$
$\frac{\partial \dot{\boldsymbol{\omega}}_{B}^{I}}{\partial \mathbf{q}}$ is however much more difficult to determine. The first step is to determine which components of Equation B.2 are dependent on the attitude quaternion of the satellite. The control torque $\mathbf{N}_{c}$ is the sum of the torques generated by the ADCS actuators, which means that $\mathbf{N}_{c}$ is independent of $\mathbf{q}$. $\mathbf{N}_{\text {gyro }}$ is calculated using only the moment of inertia matrix, the angular rates and the stored angular momentum, which means that $\mathbf{N}_{\text {gyro }}$ is also independent of $\mathbf{q}$. Although there are many sources of disturbance torques, $\mathbf{N}_{d}$ at a LEO orbit is simplified to contain only two major components, namely gravity gradient torque $\left(\mathbf{N}_{g g}\right)$ and aerodynamic torque $\left(\mathbf{N}_{a e r o}\right)$. Even though both these components are dependent on the attitude of the satellite, only $\mathbf{N}_{g g}$ can be calculated accurately, thus
$$
\mathbf{N}_{d} \approx \mathbf{N}_{g g}
$$
$\mathbf{N}_{d}$ can thus easily be expressed in terms of quaternions using Equation $2.3 .3$ as
$$
\begin{aligned}
&\mathrm{N}_{d x} \approx k_{g x}\left(2\left[q_{2} q_{3}+q_{1} q_{4}\right]\right)\left(-q_{1}^{2}-q_{2}^{2}+q_{3}^{2}+q_{4}^{2}\right) \\
&\mathrm{N}_{d y} \approx k_{g y}\left(2\left[q_{1} q_{3}-q_{2} q_{4}\right]\right)\left(-q_{1}^{2}-q_{2}^{2}+q_{3}^{2}+q_{4}^{2}\right) \\
&\mathrm{N}_{d z} \approx k_{g z}\left(2\left[q_{1} q_{3}-q_{2} q_{4}\right]\right)\left(2\left[q_{2} q_{3}+q_{1} q_{4}\right]\right)
\end{aligned}
$$
$\frac{\partial \dot{\omega}_{B}^{I}}{\partial \mathbf{q}}$ can now be calculated as
$$
\frac{\partial \dot{\boldsymbol{\omega}}_{B}^{I}}{\partial \mathbf{q}}=\mathbf{J}^{-1}\left[\frac{\partial \mathbf{N}_{d}}{\partial \mathbf{q}}\right]=\mathbf{K}\left[\begin{array}{llll}
\mathbf{d}_{1} & \mathbf{d}_{2} & \mathbf{d}_{3} & \mathbf{d}_{4}
\end{array}\right]
$$
where
$$
\mathbf{K}=\left[\begin{array}{ccc}
2 k_{g x} & 0 & 0 \\
0 & 2 k_{g y} & 0 \\
0 & 0 & 2 k_{g z}
\end{array}\right]
$$
and

$\frac{\partial \dot{\mathbf{q}}}{\partial \boldsymbol{\omega}_{B}^{I}}$ and $\frac{\partial \dot{\mathbf{q}}}{\partial \mathbf{q}}$ can be determined by partially deriving the time derivative of $\mathbf{q}$, which is

\includegraphics[max width=\textwidth]{2022_07_21_93209b3bc79104a60849g-03}

The relationship between $\omega_{B}^{I}$ and $\omega_{B}^{O}$ is given by
$$
\boldsymbol{\omega}_{B}^{O}=\boldsymbol{\omega}_{B}^{I}-\mathbf{A}\left[\begin{array}{c}
0 \\
-\omega_{o} \\
0
\end{array}\right]=\left[\begin{array}{c}
\omega_{x i}+\omega_{o} A_{12} \\
\omega_{y i}+\omega_{o} A_{22} \\
\omega_{z i}+\omega_{o} A_{32}
\end{array}\right]
$$
which means that $\frac{\partial \dot{q}}{\partial \omega_{B}^{I}}$ can be determined as
$$
\frac{\partial \dot{\mathbf{q}}}{\partial \omega_{B}^{I}}=\left[\begin{array}{ccc}
\frac{\partial \dot{q}_{1}}{\partial \omega_{x i}} & \frac{\partial \dot{q}_{1}}{\partial \omega_{y i}} & \frac{\partial \dot{q}_{1}}{\partial \omega_{z i}} \\
\frac{\partial \dot{q}_{2}}{\partial \omega_{x i}} & \frac{\partial \dot{q}_{2}}{\partial \omega_{y i}} & \frac{\partial \dot{q}_{2}}{\partial \omega_{z i}} \\
\frac{\partial \dot{q}_{3}}{\partial \omega_{x i}} & \frac{\partial \dot{q}_{3}}{\partial \omega_{y i}} & \frac{\partial \dot{q}_{3}}{\partial \omega_{z i}} \\
\frac{\partial \dot{q}_{4}}{\partial \omega_{x i}} & \frac{\partial \dot{q}_{4}}{\partial \omega_{y i}} & \frac{\partial \dot{q}_{4}}{\partial \omega_{z i}}
\end{array}\right]=\frac{1}{2}\left[\begin{array}{ccc}
\hat{q}_{4} & -\hat{q}_{3} & \hat{q}_{2} \\
\hat{q}_{3} & \hat{q}_{4} & -\hat{q}_{1} \\
-\hat{q}_{2} & \hat{q}_{1} & \hat{q}_{4} \\
-\hat{q}_{1} & -\hat{q}_{2} & -\hat{q}_{3}
\end{array}\right]
$$
$\frac{\partial \dot{\mathbf{q}}}{\partial \mathbf{q}}$ can be determined by substituting Equation B.13 into Equation B.12, which delivers
$$
\begin{aligned}
\dot{q}_{1} &=\frac{1}{2}\left(q_{2}\left(\omega_{z i}-\omega_{o} A_{32}\right)-q_{3}\left(\omega_{y i}-\omega_{o} A_{22}\right)+q_{4}\left(\omega_{x i}-\omega_{o} A_{12}\right)\right) \\
\dot{q}_{2} &=\frac{1}{2}\left(-q_{1}\left(\omega_{z i}-\omega_{o} A_{32}\right)+q_{3}\left(\omega_{x i}-\omega_{o} A_{12}\right)+q_{4}\left(\omega_{y i}-\omega_{o} A_{22}\right)\right) \\
\dot{q}_{3} &=\frac{1}{2}\left(q_{1}\left(\omega_{y i}-\omega_{o} A_{22}\right)-q_{2}\left(\omega_{x i}-\omega_{o} A_{12}\right)+q_{4}\left(\omega_{z i}-\omega_{o} A_{32}\right)\right) \\
\dot{q}_{4} &=\frac{1}{2}\left(-q_{1}\left(\omega_{x i}-\omega_{o} A_{12}\right)-q_{2}\left(\omega_{y i}-\omega_{o} A_{22}\right)-q_{3}\left(\omega_{z i}-\omega_{o} A_{32}\right)\right)
\end{aligned}
$$
$$
\begin{aligned}
& \mathbf{d}_{1}=\left[\begin{array}{c}\frac{-q_{1} A_{23}+q_{4} A_{33}}{I_{x x}} \\\frac{-q_{1} A_{13}+q_{3} A_{33}}{I_{y y}} \\\frac{q_{3} A_{23}+q_{4} A_{13}}{I_{z z}}\end{array}\right] \quad \mathbf{d}_{2}=\left[\begin{array}{l}\frac{-q_{2} A_{23}+q_{3} A_{33}}{I_{x x}} \\\frac{-q_{2} A_{13}-q_{4} A_{33}}{I_{y y}} \\\frac{-q_{4} A_{23}+q_{3} A_{13}}{I_{z z}}\end{array}\right] \\
& \mathbf{d}_{3}=\left[\begin{array}{c}\frac{q_{3} A_{23}+q_{2} A_{33}}{I_{x x}} \\\frac{q_{3} A_{13}+q_{1} A_{33}}{I_{y y}} \\\frac{q_{1} A_{23}+q_{2} A_{13}}{I_{z z}}\end{array}\right] \quad \mathbf{d}_{4}=\left[\begin{array}{c}\frac{q_{4} A_{23}+q_{1} A_{33}}{I_{x x}} \\\frac{q_{4} A_{13}-q_{2} A_{33}}{I_{y y}} \\\frac{-q_{2} A_{23}+q_{1} A_{13}}{I_{z z}}\end{array}\right] . 
\end{aligned}
$$
By partially deriving Equations B.15 to B.18 and performing some mathematical manipulation, $\frac{\partial \dot{\mathbf{q}}}{\partial \mathbf{q}}$ can be calculated as
$$
\frac{\partial \dot{\mathbf{q}}}{\partial \mathbf{q}}=\frac{1}{2}\left[\Omega\left(\boldsymbol{\omega}_{B}^{O}\right)\right]+\omega_{o}\left[\begin{array}{cccc}
\hat{q}_{1} \hat{q}_{3} & \hat{q}_{1} \hat{q}_{4} & 1-\hat{q}_{1}^{2} & -\hat{q}_{1} \hat{q}_{2} \\
\hat{q}_{2} \hat{q}_{3} & \hat{q}_{2} \hat{q}_{4} & -\hat{q}_{1} \hat{q}_{2} & 1-\hat{q}_{2}^{2} \\
-\left(1-\hat{q}_{3}^{2}\right) & \hat{q}_{3} \hat{q}_{4} & -\hat{q}_{1} \hat{q}_{3} & -\hat{q}_{2} \hat{q}_{3} \\
\hat{q}_{3} \hat{q}_{4} & -\left(1-\hat{q}_{4}^{2}\right) & -\hat{q}_{1} \hat{q}_{4} & -\hat{q}_{2} \hat{q}_{4}
\end{array}\right]
$$